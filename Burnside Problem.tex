\section{Burnside Group of Odd Exponents: Results by Arkarskaya, A. et al.}
\par
This note is entirely based on [1] and I will only rephrase the proofs to make them more readable. This note is written for a presentation I will give in Math 490.
\begin{definition}
    Let $F=\cyc{{x}_{1},\dots,{x}_{m}}$ be the free group ${F}_{m}$, where $m\ge 2$. The \hdef{free Burnside group} $B(m,n)=F/\cyc{\cyc{{x}_{1},\dots,{x}_{m}\mid{w}^{n},w\in F}}$. We say $B(m,n)$ has \hdef{rank} $m$ and \hdef{exponent} $n$.
\end{definition}
\begin{definition}
    Let ${F}_{m}$ be a free group of rank $m$, then elements of $\{{x}_{1},\dots,{x}_{m}\}\cup\{\inv{{x}_{1}},\dots,\inv{{x}_{m}}\}$ are called \hdef{words}. A sequence of words is called a \hdef{letter}. A word without cancellation is called a \hdef{reduced word}.
\end{definition}
\begin{definition}
    We say a word $A$ is \hdef{cyclically contained} in a word $w$ if $A$ is a subword of any cyclic shift of $w$.
\end{definition}
\begin{example}
    Let $w=abcde$ and let $A=dea$. Shift $w$ cyclically, we obtain $bcdea$, hence $A$ is cyclically contained in $w$.
\end{example}

\newpage


\begin{theorem}
    Induct on $i$, then the following properties hold.
    \begin{enumerate}
        \item ${\Can}_{i}\sub{\Can}_{i-1}$;
        \item if ${L}_{1}{A}^{\tau}{R}_{1},{L}_{2}{A}^{\tau}{R}_{2}\in{\Can}_{i}$ for some primitive $A$ and ${A}^{n}\notin$
    \end{enumerate}
\end{theorem}




\par
If the induction works, we will have the following results immediately.
\begin{theorem}
    
\end{theorem}